\documentclass[aspectratio=169]{beamer}
\mode<presentation>
%\usetheme{Warsaw}
%\usetheme{Goettingen}
\usetheme{Hannover}
%\useoutertheme{default}

%\useoutertheme{infolines}
\useoutertheme{sidebar}
\usecolortheme{dolphin}

\usepackage{amsmath}
\usepackage{amssymb}
\usepackage{enumerate}

%some bold math symbosl
\newcommand{\Cov}{\mathrm{Cov}}
\newcommand{\Cor}{\mathrm{Cor}}
\newcommand{\Var}{\mathrm{Var}}
\newcommand{\brho}{\boldsymbol{\rho}}
\newcommand{\bSigma}{\boldsymbol{\Sigma}}
\newcommand{\btheta}{\boldsymbol{\theta}}
\newcommand{\bbeta}{\boldsymbol{\beta}}
\newcommand{\bmu}{\boldsymbol{\mu}}
\newcommand{\bW}{\mathbf{W}}
\newcommand{\one}{\mathbf{1}}
\newcommand{\bH}{\mathbf{H}}
\newcommand{\by}{\mathbf{y}}
\newcommand{\bolde}{\mathbf{e}}
\newcommand{\bx}{\mathbf{x}}

\newcommand{\cpp}[1]{\texttt{#1}}

 \title{Mathematical Biostatistics Boot Camp 2: Lecture }
\author{Brian Caffo}
\date{\today}
\institute[Department of Biostatistics]{
  Department of Biostatistics \\
  Johns Hopkins Bloomberg School of Public Health\\
  Johns Hopkins University
}


\begin{document}
\frame{\titlepage}

%\section{Table of contents}
\frame{
  \frametitle{Table of contents}
  \tableofcontents
}

\begin{frame}\frametitle{Pump failure data}
\begin{center}
\ttfamily
\begin{tabular}{llllll} 
Pump     & 1     & 2     & 3     & 4      & 5     \\ \hline
Failures & 5     & 1     & 5     & 14     & 3     \\ 
Time     & 94.32 & 15.72 & 62.88 & 125.76 & 5.24  \\ \hline \\
Pump     & 6     & 7     & 8     & 9      & 10    \\ \hline
Failures & 19    & 1     & 1     & 4      & 22    \\ 
Time     & 31.44 & 1.05  & 1.05  & 2.10   & 10.48 \\ \hline 
\end{tabular}
\end{center}
From Casella and Robert, Monte Carlo Statistical Methods; first edition
\normalfont \normalsize
\end{frame}

\section{The Poisson distribution}
\begin{frame}\frametitle{The Poisson distribution}
\begin{itemize}
\item Used to model counts
\item The Poisson mass function is
$$
P(X = x; \lambda) = \frac{\lambda^x e^{-\lambda}}{x!}
$$
for $x=0,1,\ldots$
\item The mean of this distribution is $\lambda$
\item The variance of this distribution is $\lambda$
\item Notice that $x$ ranges from $0$ to $\infty$
\end{itemize}
\end{frame}

\begin{frame}\frametitle{Some uses for the Poisson distribution}
\begin{itemize}
\item Modeling event/time data
\item Modeling radioactive decay
\item Modeling survival data
\item Modeling unbounded count data 
\item Modeling contingency tables
\item Approximating binomials when $n$ is large and $p$ is small
\end{itemize}
\end{frame}

\begin{frame}\frametitle{Definition}
\begin{itemize}
\item $\lambda$ is the mean number of events per unit time
\item Let $h$ be very small 
\item Suppose we assume that 
  \begin{itemize}
  \item Prob. of an event in an interval of length $h$ is $\lambda h$
    while the prob. of more than one event is negligible
  \item Whether or not an event occurs in one small interval
    does not impact whether or not an event occurs in another
    small interval
  \end{itemize}
then, the number of events per unit time is Poisson with mean $\lambda$ 
\end{itemize}
\end{frame}

\section{Poisson approximation to the binomial}
\begin{frame}\frametitle{Poisson approximation to the binomial}
\begin{itemize}
\item When $n$ is large and $p$ is small the Poisson distribution
  is an accurate approximation to the binomial distribution
\item Notation
  \begin{itemize}
  \item $\lambda = n p$
  \item $X \sim \mbox{Binomial}(n, p)$, $\lambda = n p$ and
  \item $n$ gets large 
  \item $p$ gets small
  \item $\lambda$ stays constant
  \end{itemize}
\end{itemize}
\end{frame}

\begin{frame}\frametitle{Proof} Rice Mathematical Statistics and Data Analysis page 41
\begin{eqnarray*}
  P(X = k) & = & \frac{n!}{k!(n-k)!} p^k (1 - p)^{n-k} \\ \\
           & = & \frac{n!}{k!(n-k)!} \left(\frac{\lambda}{n}\right)^k\left(1 - \frac{\lambda}{n}\right)^{n-k} \\ \\
           & = & \frac{n!}{(n-k)!n^k} \times \left(1 - \frac{\lambda}{n}\right)^{-k} \times \frac{\lambda^k}{k!}\left(1 - \frac{\lambda}{n}\right)^n \\ \\
           & \rightarrow & 1 \times 1 \times \frac{\lambda^k}{k!} e^{-\lambda} \\ \\
           & = & \frac{\lambda^k}{k!} e^{-\lambda}
\end{eqnarray*}
\end{frame}

 \begin{frame}\frametitle{Notes}
 \begin{itemize}
 \item That $(1 - \lambda/n)^n$ converges to $e^{-\lambda}$ for large $n$ is 
   a very old mathematical fact
 \item We can show that $\frac{n!}{(n-k)!n^k}$ goes to one easily because
   $$
   \frac{n!}{(n-k)!n^k} = 1 \times \left(1 - \frac{1}{n}\right) \times \left(1 - \frac{2}{n}\right) \times \ldots
   \times \left(1 - \frac{k - 1}{n}\right)
   $$
   each term goes to 1
\end{itemize}
\end{frame}

\begin{frame}\frametitle{Examples}
Some example uses of the Poisson distribution
\begin{itemize}
\item deaths per day in a city 
\item homicides witnessed in a year
\item teen pregnancies per month
\item Medicare claims per day
\item cases of a disease per year
\item cars passing an intersection in a day
\item telephone calls received by a switchboard in an hour
\end{itemize}
\end{frame}

\begin{frame}\frametitle{Some results}
\begin{itemize}
\item If $X \sim \mbox{Poisson}(t\lambda)$ then 
$$
\frac{X - t\lambda}{\sqrt{t\lambda}} = \frac{X - \mbox{Mean}}{\mbox{SD}} 
$$
converges to a standard normal as $t\lambda \rightarrow \infty$
\item Hence 
$$
\frac{(X - t\lambda)^2}{t\lambda} = \frac{(O - E)^2}{E} 
$$
converges to a Chi-squared with 1 degree of freedom for large $t\lambda$
\item If $X \sim \mbox{Poisson}(t\lambda)$ then $X/t$ is the ML estimate
  of $\lambda$
\end{itemize}
\end{frame}

\begin{frame}\frametitle{Proof}
\begin{itemize}
\item Likelihood is
$$
\frac{(t\lambda)^x e^{-t\lambda}}{x!}
$$
\item So that the log-likelihood is
$$
x\log(\lambda) - t\lambda + \mbox{constants in } \lambda
$$
\item The derivative of the log likelihood is
$$
x / \lambda - t
$$
\item Setting equal to $0$ we get that $\hat \lambda = x / t$
\end{itemize}
\end{frame}

\begin{frame}\frametitle{Pump failure data}
\begin{itemize}
\item Failures for Pump 1: $5$, monitoring time: $94.32$ days
\item Estimate of $\lambda$, the mean number of failures per day $=5 / 94.32 = .053$
\item Test the hypothesis that the mean number of failures per day is larger
  than the industry standard, $.15$ events per day:
  $H_0 :\lambda = .15$ versus $H_a : \lambda > .15$ 
\item $TS = (5 - 94.32 \times .15) / \sqrt{94.2 \times .15} = -2.433$
\item Hence P-value is very large $(.99)$
\item HW: Obtain a confidence interval for $\lambda$ 
\end{itemize}
\end{frame}

\begin{frame}[fragile]\frametitle{Pump failure data}
\begin{itemize}
\item  Exact P-value can be obtained by using the Poisson distribution directly
\item  $P(X \geq 5)$ where $X \sim \mbox{Poisson}(.15 \times 94.32)$
\begin{verbatim}
ppois(5, .15 * 94.32, lower.tail = FALSE) = .995
\end{verbatim}
very little evidence to suggest that this pump is malfunctioning
\item To obtain a P-value for a two-sided alternative, double the
  smaller of the two one sided P-values
\end{itemize}
\end{frame}

\begin{frame}
\begin{center}
\ttfamily
\begin{tabular}{llllll} 
Pump           & 1      & 2     & 3     & 4      & 5     \\ \hline
Failures       & 5      & 1     & 5     & 14     & 3     \\ 
Time           & 94.32  & 15.72 & 62.88 & 125.76 & 5.24  \\ 
$\hat \lambda$ & .053   & .064  & .080  & .111   & .573  \\ 
P-value        & .995   & .999  & .908  & .843   & .009  \\ \\ \hline
Pump           & 6      & 7     & 8     & 9      & 10    \\ \hline
Failures       & 19     & 1     & 1     & 4      & 22    \\ 
Time           & 31.44  & 1.05  & 1.05  & 2.10   & 10.48 \\ 
$\hat \lambda$ & .604   & .952  & .952  & 1.904  & 2.099 \\ 
P-value        & 1e-7   & .011  & .011  & 1e-5   & 2e-19 \\ \hline
\end{tabular}
\end{center}
\normalfont  \normalsize
\end{frame}

\begin{frame}\frametitle{Common failure rate}
\begin{itemize}
\item If $X_i$ for $i=1,\ldots,n$ are Poisson$(t_i\lambda)$ then 
  $$
  \sum X_i \sim \mbox{Poisson}\left(\lambda\sum t_i \right)
  $$
  If you are willing to assume the common $\lambda$, then $\sum X_i$ contains
  all of the relevant information
\item Clearly a common $\lambda$ across pumps is not warranted. However, for illustration,
  assume that this is the case
\item Then the total number of failures was $75$
\item The total monitoring time was $305.4$
\item The estimate of the common $\lambda$ would be $75 / 305.4 = .246$
\item Later we will discuss a test for common failure rate
\end{itemize}
\end{frame}

\section{Person-time analysis}
\begin{frame}\frametitle{Person-time analysis}
\begin{center}
  \ttfamily
  \begin{tabular}{lccc}
    OC        & \# of cases & ~~&\# of person-years \\ \hline
Current users & 9         & & 2,935        \\
Never   users & 239       & & 135,130      \\ \hline
  \end{tabular}\\
From: Rosner Fundamentals of Biostatistics, sixth edition, page 744
\end{center}
\begin{itemize}
\item One of the most common uses of the Poisson distribution in 
  epi is to model rates
\item Estimated incidence rate amongst current users is $9/2,935 = .0038$
  events per person year
\item Estimated incidence rate amongst never users is $239 / 135,130 =
  .0018$ events per person year
\end{itemize}
\end{frame}

\begin{frame}\frametitle{Poisson model}
\begin{itemize}
\item Rates are often {\em modeled} as Poisson
\item Notice that the total number of subjects is discarded, whether
  the 2,935 years was comprised of 1,000 or 500 people does not come
  into play
\item Most useful for rare events (though we'll discuss another motivation for the
  Poisson model later)
  \begin{itemize}
  \item $\lambda = .3$ events per year
  \item Followed 10 people for a total of 40 person-years
  \item Expected number of deaths is $.3\times 40 = 12$, larger than our sample
  \end{itemize}
\item $\lambda$ is assumed constant over time
\end{itemize}
\end{frame}

\begin{frame}\frametitle{Comparing two Poisson means}
\begin{itemize}
\item Want to test $H_0 : \lambda_1 = \lambda_2 = \lambda$ 
\item Observed counts $x_1$, $x_2$ Person-times $t_1$, $t_2$
\item Estimate of $\lambda$ under the null hypothesis is 
  $\hat \lambda = \frac{x_1 + x_2}{t_1 + t_2}$
\item Estimated expected count in Group 1 under $H_0$
  $$
  E_1 = \hat \lambda t_1 = (x_1 + x_2) \frac{t_1}{t_1 + t_2} 
  $$
\item Estimated expected count in Group 2 under $H_0$
  $$
  E_2 = \hat \lambda t_2 = (x_1 + x_2) \frac{t_2}{t_1 + t_2} 
  $$
\end{itemize}
\end{frame}

\begin{frame}\frametitle{Notes}
\begin{itemize}
\item Test statistic
  $$
  TS = \sum \frac{(O - E)^2}{E} =  \frac{(x_1 - E_1)^2}{E_1} + \frac{(x_2 - E_2)^2}{E_2}
  $$
follows a Chi-squared distribution with 1 df
\item Equivalent computational form
  $$
  TS = \frac{(X_1 - E_1)^2}{V_1}
  $$
  where 
  $$
  V_1 = (x_1 + x_2)t_1 t_2 / (t_1 + t_2)^2
  $$
\end{itemize}
\end{frame}

\begin{frame}\frametitle{OC example}
\begin{itemize}
\item $x_1 = 9$, $t_1 = 2,935$
\item $x_2 = 239$, $t_2 = 135,130$
\item $E_1 = (9 + 239) \times \frac{2,935}{2,935 + 135,130} = 5.27$
\item $V_1 = (9 + 239) \times \frac{2,935 \times 135,130}{(2,935 + 135,130)^2} = 5.16$
$$
TS = \frac{(9 - 5.27)^2}{5.16} = 2.70
$$
P-value = $.100$
\end{itemize}
\end{frame}

\begin{frame}\frametitle{Estimating the relative rate}
\begin{itemize}
\item Relative rate $\lambda_1 / \lambda_2$
\item Estimate $(x_1 / t_1) / (x_2 / t_2)$
\item Standard error for the {\bf log} relative rate estimate
  $$
\sqrt{  \frac{1}{x_1} + \frac{1}{x_2}}
  $$
\item For the OC example the estimated log relative rate
  is $.550$
\item The standard error is $\sqrt{\frac{1}{9} + \frac{1}{239}} = .340$
\item 95\% CI for the log relative rate is
  $$
  .550 \pm 1.96 \times .340 = (-.115, 1.26)
  $$
\end{itemize}
\end{frame}

\section{Exact tests}
\begin{frame}\frametitle{Exact test}
  \begin{itemize}
  \item HW: Extend the above test to more than two groups
  \item Suppose that the observed counts were low
  \item We obtain an exact test by conditioning on the sum of the counts
  \item That is, we make use of the fact that
    $$
    X_1 ~|~ X_1 + X_2 = z ~\sim~\mbox{Binomial}\left(\frac{t_1}{t_1+t_2},z\right)
    $$
  \end{itemize}  
\end{frame}

\section{Time-to-event modeling}
\begin{frame}\frametitle{Alternate motivation for Poisson model}
\begin{itemize}
\item Another way to motivate the Poisson model is to show that
  it is {\em likelihood equivalent} to a plausible model
\item We show that the Poisson model is likelihood equivalent to
  a model that specifies failure times as being independent exponentials
\item Whenever the independent exponential model is reasonable, then so
  is the Poisson model, regardless of how large or small the rate or
  sample size is
\item The likelihood equivalence implies that likelihood methods apply
\end{itemize}
\end{frame}

\begin{frame}\frametitle{Recall}
\begin{itemize}
\item The likelihood is the density viewed as a function of the parameter
\item The likelihood summarizes the evidence in the data about the parameter
\item If $X \sim \mbox{Poisson}(t\lambda)$ then 
  the likelihood for $\lambda$ is 
  $$
  {\cal L}(\lambda) = \frac{(t\lambda)^x e^{-t\lambda}}{x!} \propto \lambda^x e^{-t\lambda}
  $$
\item When you have independent observations, the likelihood is the product
  of the likelihood for each observation
\end{itemize}

\end{frame}

\begin{frame}\frametitle{Likelihood equivalence with exponential model}
\begin{itemize}
\item Suppose there is no censoring (every person followed until an event)
\item We model each person's time until failure as independent exponentials:
$Y_i \sim \mbox{Exponential}(\lambda)$
\item Each subject contributes $\lambda e^{-y_i \lambda}$ to the likelihood
  $$
  \prod_{i=1}^n \lambda e^{-y_i\lambda} = \lambda^n \exp\left(-\lambda\sum y_i\right)
  $$
\item Here $n$ is the number of events and $\sum y_i$ is the total person-time
\item Same likelihood as if we specify $n \sim \mbox{Poisson}(\lambda \sum y_i)$!
\end{itemize}

\end{frame}

\begin{frame}\frametitle{Likelihood equivalence with censoring}
\begin{itemize}
\item Suppose now that the study ended before events were observed on some subjects
\item Likelihood contribution for each event is $\lambda e^{-y_i \lambda}$
\item Likelihood contribution for each censored observation is 
  $$
  P(Y \geq y_i; \lambda) = \int_{y_i}^\infty \lambda e^{-u \lambda}du = e^{-y_i \lambda}
  $$
  The contribution is this integral b/c we don't really know the event time for that
  person, only that it was later than $y_i$
\end{itemize}
\end{frame}
 
\begin{frame}
\begin{itemize}
\item Combining the censored and non-censored observations, we have
$$
\left\{\prod_{\mathrm{non-censored} ~i}\lambda e^{-y_i \lambda}\right\}\times
\left\{\prod_{\mathrm{censored} ~i} e^{-y_i \lambda}\right\}
$$
$x$ events (and hence $n - x$ censored observations)
$$
= \lambda^x \exp\left(-\lambda \sum_{i=1}^n y_i\right)
$$
\item $x$ is the total number of events and $\sum_{i=1}^n y_i$ is the total amount of
  person-time
\item This is exactly the same likelihood as modeling $X \sim \mbox{Poisson}(\lambda \sum_{i=1}^n y_i)$
\end{itemize}
\end{frame}

\end{document}
