\documentclass[aspectratio=169]{beamer}
\mode<presentation>
%\usetheme{Warsaw}
%\usetheme{Goettingen}
\usetheme{Hannover}
%\useoutertheme{default}

%\useoutertheme{infolines}
\useoutertheme{sidebar}
\usecolortheme{dolphin}

\usepackage{amsmath}
\usepackage{amssymb}
\usepackage{enumerate}

%some bold math symbosl
\newcommand{\Cov}{\mathrm{Cov}}
\newcommand{\Cor}{\mathrm{Cor}}
\newcommand{\Var}{\mathrm{Var}}
\newcommand{\brho}{\boldsymbol{\rho}}
\newcommand{\bSigma}{\boldsymbol{\Sigma}}
\newcommand{\btheta}{\boldsymbol{\theta}}
\newcommand{\bbeta}{\boldsymbol{\beta}}
\newcommand{\bmu}{\boldsymbol{\mu}}
\newcommand{\bW}{\mathbf{W}}
\newcommand{\one}{\mathbf{1}}
\newcommand{\bH}{\mathbf{H}}
\newcommand{\by}{\mathbf{y}}
\newcommand{\bolde}{\mathbf{e}}
\newcommand{\bx}{\mathbf{x}}

\newcommand{\cpp}[1]{\texttt{#1}}

 \title{Mathematical Biostatistics Boot Camp 2: Lecture 7, Fisher's Exact Test}
\author{Brian Caffo}
\date{\today}
\institute[Department of Biostatistics]{
  Department of Biostatistics \\
  Johns Hopkins Bloomberg School of Public Health\\
  Johns Hopkins University
}


\begin{document}
\frame{\titlepage}

%\section{Table of contents}
\frame{
  \frametitle{Table of contents}
  \tableofcontents
}

\section{Fisher's exact test}
\begin{frame}\frametitle{Fisher's exact test}
\begin{itemize}
\item Fisher's exact test is ``exact'' because it guarantees the $\alpha$ rate, 
  regardless of the sample size
\item Example, chemical toxicant and 10 mice
\begin{center}
  \begin{tabular}{lccl}
        & Tumor & None     & Total \\ \hline
Treated & 4     & 1        & 5 \\ 
Control & 2     & 3        & 5 \\ \hline
Total   & 6     & 4        &
  \end{tabular}
\end{center}
\item $p_1 = $ prob of a tumor for the treated mice
\item $p_2 = $ prob of a tumor for the untreated mice
\end{itemize}
\end{frame}

\begin{frame}\frametitle{Continued}
  \begin{itemize}
  \item $H_0:p_1 = p_2 = p$
  \item Can't use $Z$ or $\chi^2$ because SS is small
  \item Don't have a specific value for $p$
  \end{itemize}
\end{frame}

\begin{frame}[fragile]\frametitle{Fisher's exact test}
  \begin{itemize}
  \item Under the null hypothesis every permutation is equally likely
  \item observed data
\begin{verbatim}
Treatment : T T T T T C C C C C         
Tumor     : T T T T N T T N N N 
\end{verbatim}
  \item permuted
\begin{verbatim}
Treatment : T C C T C T T C T C         
Tumor     : T T T T N T T N N N 
\end{verbatim}
  \item Fisher's exact test uses this null distribution to test the
    hypothesis that $p_1 = p_2$
  \end{itemize}
\end{frame}

\section{The hypergeometric distribution}
\begin{frame}\frametitle{Hyper-geometric distribution}
  \begin{itemize}
  \item $X$ number of tumors for the treated
  \item $Y$ number of tumors for the controls
  \item $H_0: p_1 = p_2 = p$
  \item Under $H_0$
    \begin{itemize}
    \item  $X \sim \mbox{Binom}(n_1, p)$
    \item  $Y \sim \mbox{Binom}(n_2, p)$
    \item  $X + Y \sim \mbox{Binom}(n_1 + n_2, p)$
    \end{itemize}
\end{itemize}
\end{frame}

\begin{frame}\frametitle{Continued}
$$P(X = x ~|~ X + Y = z) = 
\frac{
\left(
\begin{array}{c}
n_1 \\ x
\end{array}
\right)
\left(
\begin{array}{c}
n_2 \\ z - x
\end{array}
\right)
}
{
\left(
\begin{array}{c}
n_1 + n_2 \\ z
\end{array}
\right)
}
$$
This is the hypergeometric pmf
\end{frame}

\begin{frame}\frametitle{Proof}
$$
P(X = x) = \left( \begin{array}{c} n_1 \\ x \end{array} \right) p^x(1-p)^{n_1 - x}
$$ \ \\
$$
P(Y = z-x) = \left( \begin{array}{c} n_2 \\ z-x \end{array} \right) p^{z-x}(1-p)^{n_2 - z + x}
$$ \ \\
$$
P(X + Y = z) = \left( \begin{array}{c} n_1 + n_2 \\ z \end{array} \right) p^z(1-p)^{n_1+n_2 -z}
$$ 
\end{frame}
\begin{frame}\frametitle{Continued}
\begin{eqnarray*}
P(X = x ~ | ~ X + Y = z) & = &\frac{P(X = x, X + Y = z)}{P(X + Y = z)} \\
                         &   & \\
                         & = &\frac{P(X = x, Y = z - x)}{P(X + Y = z)} \\
                         &   & \\
                         & = &\frac{P(X = x)P(Y = z-x)}{P(X + Y = z)}
\end{eqnarray*}
Plug in and finish off yourselves
\end{frame}


\section{Fisher's exact test in practice}
\begin{frame}\frametitle{Fisher's exact test}
  \begin{itemize}
  \item More tumors under the treated than the controls
  \item Calculate an {\em exact} P-value 
  \item Use the conditional distribution = hypergeometric
  \item Fixes both the row and the column totals
  \item Yields the same test regardless of whether the
    rows or columns are fixed
  \item Hypergeometric distribution is the same as the
    permutation distribution given before
\end{itemize}
\end{frame}

\begin{frame}\frametitle{Tables supporting $H_a$}
\begin{itemize}
\item Consider $H_a : p_1 > p_2$
\item P-value requires tables as extreme or more extreme
  (under $H_a$) than the one observed
\item Recall we are fixing the row and column totals
\item Observed table
\begin{center}
Table 1 = 
  \begin{tabular}{cc|l} 
4     & 1        & 5 \\  
2     & 3        & 5 \\ \hline
6     & 4        &
  \end{tabular}
\end{center}
\item More extreme tables in favor of the alternative
\begin{center}
Table 2 = 
  \begin{tabular}{cc|l}
5     & 0       & 5 \\ 
1     & 4       & 5 \\\hline 
6     & 4       &
  \end{tabular} 
\end{center}
\end{itemize}
\end{frame}


\begin{frame}\frametitle{Calculations}
\begin{eqnarray*}
P(\mbox{Table 1}) & = & P(X = 4 | X + Y = 6) \\ & = &
\frac{\left(\begin{array}{c} 5 \\ 4 \end{array}\right) 
\left(\begin{array}{c} 5 \\ 2 \end{array} \right)}
{\left( \begin{array}{c} 10 \\ 6 \end{array} \right)} = 0.238
\end{eqnarray*}
\begin{eqnarray*}
P(\mbox{Table 2}) & = & P(X = 5 | X + Y = 6) \\ & = &
\frac{\left(\begin{array}{c} 5 \\ 5 \end{array}\right) 
\left(\begin{array}{c} 5 \\ 1 \end{array} \right)}
{\left( \begin{array}{c} 10 \\ 6 \end{array} \right)} = 0.024
\end{eqnarray*}
P-value = 0.238 + 0.024 = 0.262
\end{frame}

\begin{frame}[fragile]\frametitle{R code}
\begin{verbatim}
dat <- matrix(c(4, 1, 2, 3), 2)
fisher.test(dat, alternative = "greater")

------------------output----------------
        Fisher's Exact Test for Count Data

data:  dat 
p-value = 0.2619
alt hypoth: true odds ratio  is greater than 1 
95 percent confidence interval:
 0.3152217       Inf 
sample estimates:
odds ratio 
  4.918388 
\end{verbatim}
\end{frame}

\begin{frame}\frametitle{Notes}
\begin{itemize}
\item Two sided p-value = 2$\times$one sided P-value \\
  (There are other methods which we will not discuss)
\item P-values are usually large for small $n$ 
\item Doesn't distinguish between rows or columns 
\item The common value of $p$ under the null hypothesis
  is called a nuisance parameter
\item Conditioning on the total number of successes, $X + Y$, eliminates
  the nuisance parameter, $p$
\item Fisher's exact test guarantees the type I error rate
\item Exact unconditional P-value
$$
\sup_p P(X/n_1 > Y/n_2; p)
$$
\end{itemize}
\end{frame}

\section{Monte Carlo}
\begin{frame}[fragile]\frametitle{Monte Carlo}
\begin{itemize}
\item Observed table $X = 4$
\begin{verbatim} 
Treatment : T T T T T C C C C C         
Tumor     : T T T T N T T N N N 
\end{verbatim}
\item Permute the first row
\begin{verbatim}
Treatment : T C T T C C C T T T     
Tumor     : T T T T N T T N N N
\end{verbatim}
\item Simulated table $X = 3$ 
\item Do over and over 
\item Calculate the proportion of tables for 
which the simulated $X \geq 4$ 
\item This proportion is a Monte Carlo estimate
for Fisher's exact P-value
\end{itemize}
\end{frame}

\end{document}
