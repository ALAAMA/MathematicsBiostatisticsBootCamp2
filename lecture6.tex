\documentclass[aspectratio=169]{beamer}
\mode<presentation>
%\usetheme{Warsaw}
%\usetheme{Goettingen}
\usetheme{Hannover}
%\useoutertheme{default}

%\useoutertheme{infolines}
\useoutertheme{sidebar}
\usecolortheme{dolphin}

\usepackage{amsmath}
\usepackage{amssymb}
\usepackage{enumerate}

%some bold math symbosl
\newcommand{\Cov}{\mathrm{Cov}}
\newcommand{\Cor}{\mathrm{Cor}}
\newcommand{\Var}{\mathrm{Var}}
\newcommand{\brho}{\boldsymbol{\rho}}
\newcommand{\bSigma}{\boldsymbol{\Sigma}}
\newcommand{\btheta}{\boldsymbol{\theta}}
\newcommand{\bbeta}{\boldsymbol{\beta}}
\newcommand{\bmu}{\boldsymbol{\mu}}
\newcommand{\bW}{\mathbf{W}}
\newcommand{\one}{\mathbf{1}}
\newcommand{\bH}{\mathbf{H}}
\newcommand{\by}{\mathbf{y}}
\newcommand{\bolde}{\mathbf{e}}
\newcommand{\bx}{\mathbf{x}}

\newcommand{\cpp}[1]{\texttt{#1}}

 \title{Mathematical Biostatistics Boot Camp 2: Lecture 6, Delta Method}
\author{Brian Caffo}
\date{\today}
\institute[Department of Biostatistics]{
  Department of Biostatistics \\
  Johns Hopkins Bloomberg School of Public Health\\
  Johns Hopkins University
}


\begin{document}
\frame{\titlepage}

%\section{Table of contents}
\frame{
  \frametitle{Table of contents}
  \tableofcontents
}

\section{Recap}
\begin{frame}\frametitle{Two sample binomials results} 
  Recall $X \sim \mathrm{Bin}(n_1, p_1)$ and $Y\sim \mathrm{Bin}(n_2, p_2)$. Also this
  information is often arranged in a 2$\times$2 table:
  \begin{center}
    \begin{tabular}{|c|c|c|}\hline
      $n_{11} = x$ & $n_{12} = n_1 - x$ & $n_1$ \\ \hline
      $n_{21} = y$ & $n_{22} = n_2 - y$ & $n_2$ \\ \hline
    \end{tabular}
  \end{center}
  \begin{itemize}
  \item $\hat{RD} = \hat p_1 - \hat p_2$ \\ \ \\
    $\hat{SE}_{\hat{RD}} = \sqrt{\frac{\hat p_1 (1 - \hat p_1)}{n_1} + \frac{\hat p_2(1 - \hat p_2)}{n_2}}$
  \item $\hat{RR} = \frac{\hat p_1}{\hat p_2}$ \\ \ \\
 $\hat{SE}_{\log \hat{RR}} = \sqrt{\frac{(1 - \hat p_1)}{\hat p_1 n_1} + \frac{(1 - \hat p_2)}{\hat p_2 n_1}}$ 
\item $\hat{OR} = \frac{\hat p_1 / (1 - \hat p_1)}{\hat p_2 / (1 - \hat p_2)} = \frac{n_{11} n_{22}}{n_{12}n_{21}}$ \\ \ \\
  $\hat{SE}_{\log \hat{OR}} = \sqrt{\frac{1}{n_{11}} + \frac{1}{n_{12}} + \frac{1}{n_{21}} + \frac{1}{n_{22}}}$
\end{itemize}
\ \\
$\mbox{CI} = \mbox{Estimate} \pm Z_{1 - \alpha/2} \mbox{SE}_{\mbox{Est}}$ 
\end{frame}

\section{The delta method}
\begin{frame}\frametitle{Standard errors}
\begin{itemize}
\item {\bf delta method} can be used to obtain large sample standard errors
\item Formally, the delta methods states that if 
  $$
  \frac{\hat \theta - \theta}{\hat{SE}_{\hat \theta}} \rightarrow \mathrm{N}(0, 1)
  $$
  then
  $$
  \frac{f(\hat \theta) - f(\theta)}{f'(\hat \theta) \hat{SE}_{\hat \theta}}
  \rightarrow \mathrm{N}(0, 1)
  $$
\item Asymptotic mean of $f(\hat \theta)$ is $f(\theta)$
\item Asymptotic standard error of $f(\hat \theta)$ can be
  estimated with $f'(\hat \theta) \hat{SE}_{\hat \theta}$
\end{itemize}
\end{frame}

\begin{frame}\frametitle{Example}
  \begin{itemize}
  \item $\theta = p_1$
  \item $\hat \theta = \hat p_1$
  \item $\hat{SE}_{\hat \theta} = \sqrt{\frac{\hat p_1 (1 - \hat p_1)}{n_1}}$
  \item $f(x) = \log(x)$ 
  \item $f'(x) = 1 / x$
  \item $\frac{\hat \theta - \theta}{\hat{SE}_{\hat \theta}} \rightarrow \mathrm{N}(0,1)$ by the CLT 
  \item Then $\hat{SE}_{\log \hat p_1} = f'(\hat \theta) \hat{SE}_{\hat \theta}$
    $$ = \frac{1}{\hat p_1}  \sqrt{\frac{\hat p_1 (1 - \hat p_1)}{ n_1}} = \sqrt{\frac{(1 - \hat p_1)}{\hat p_1 n_1}}$$
  \item And 
    $$
    \frac{\log \hat p_1 - \log p_1}{ \sqrt{\frac{(1 - \hat p_1)}{\hat p_1 n_1}}} 
    \rightarrow \mathrm{N}(0,1)
    $$
  \end{itemize}
\end{frame}

\begin{frame}\frametitle{Putting it all together}
  \begin{itemize}
  \item Asymptotic standard error
    \begin{eqnarray*}
      \Var(\log \hat RR)
      & = & \Var\{\log(\hat p_1 / \hat p_2)\} \\
      & = & \Var(\log \hat p_1) + \Var(\log \hat p_2)  \\
      & \approx & \frac{(1 - \hat p_1)}{\hat p_1 n_1} + \frac{(1 - \hat p_2)}{ \hat p_2 n_2}
    \end{eqnarray*}
  \item The last line following from the delta method
  \item The approximation requires large sample sizes
  \item The delta method can be used similarly for the log odds ratio
  \end{itemize}
\end{frame}

\section{Derivation of the delta method}
\begin{frame}\frametitle{Motivation for the delta method}
  \begin{itemize}
  \item If $\hat \theta$ is close to $\theta$ then
    $$
    \frac{f(\hat \theta) - f(\theta)}{\hat \theta - \theta} \approx f'(\hat\theta) 
    $$
  \item So
    $$
    \frac{f(\hat \theta) - f(\theta)}{f'(\hat \theta)} \approx \hat \theta - \theta
    $$
  \item Therefore
    $$
    \frac{f(\hat \theta) - f(\theta)}{f'(\hat \theta)\hat{SE}_{\hat \theta}} \approx  
    \frac{\hat \theta - \theta}{\hat{SE}_{\hat \theta}}
    $$
  \end{itemize}
\end{frame}

\end{document}
