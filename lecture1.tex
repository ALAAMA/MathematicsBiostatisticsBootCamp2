\documentclass[aspectratio=169]{beamer}
\mode<presentation>
%\usetheme{Warsaw}
%\usetheme{Goettingen}
\usetheme{Hannover}
%\useoutertheme{default}

%\useoutertheme{infolines}
\useoutertheme{sidebar}
\usecolortheme{dolphin}

\usepackage{amsmath}
\usepackage{amssymb}
\usepackage{enumerate}
%test
%some bold math symbosl
\newcommand{\Cov}{\mathrm{Cov}}
\newcommand{\Var}{\mathrm{Var}}
\newcommand{\brho}{\boldsymbol{\rho}}
\newcommand{\bSigma}{\boldsymbol{\Sigma}}
\newcommand{\btheta}{\boldsymbol{\theta}}
\newcommand{\bbeta}{\boldsymbol{\beta}}
\newcommand{\bmu}{\boldsymbol{\mu}}
\newcommand{\bW}{\mathbf{W}}
\newcommand{\one}{\mathbf{1}}
\newcommand{\bH}{\mathbf{H}}
\newcommand{\by}{\mathbf{y}}
\newcommand{\bolde}{\mathbf{e}}
\newcommand{\bx}{\mathbf{x}}

\newcommand{\cpp}[1]{\texttt{#1}}

\title{Mathematical Biostatistics Boot Camp 2: Lecture 1, Hypothesis Testing}
\author{Brian Caffo}
\date{\today}
\institute[Department of Biostatistics]{
  Department of Biostatistics \\
  Johns Hopkins Bloomberg School of Public Health\\
  Johns Hopkins University
}


%\logo{\includegraphics[height=0.5cm]{Logo_PPT.pdf}}

\begin{document}
\frame{\titlepage}

%\section{Table of contents}
\frame{
  \frametitle{Table of contents}
  \tableofcontents
}


\section{Hypothesis testing}
\begin{frame}\frametitle{Hypothesis Testing}
\begin{itemize}
\item Hypothesis testing is concerned with making decisions using data
\item A null hypothesis is specified that represents the status quo,
  usually labeled $H_0$
\item The null hypothesis is assumed true and statistical evidence is required
  to reject it in favor of a research or alternative hypothesis 
\end{itemize}
\end{frame}

\begin{frame}\frametitle{Example}
\begin{itemize}
\item A respiratory disturbance index of more than $30$ events / hour, say, is 
  considered evidence of severe sleep disordered breathing (SDB).
\item Suppose that in a sample of $100$ overweight subjects with other
  risk factors for sleep disordered breathing at a sleep clinic, the
  mean RDI was $32$ events / hour with a standard deviation of $10$ events / hour.
\item We might want to test the hypothesis that 
  \begin{itemize}
  \item $H_0 : \mu = 30$
  \item $H_a : \mu > 30$
  \end{itemize}
  where $\mu$ is the population mean RDI.
\end{itemize}
\end{frame}

\begin{frame}\frametitle{Hypothesis testing}
\begin{itemize}
\item The alternative hypotheses are typically of the form $<$, $>$ or $\neq$
\item Note that there are four possible outcomes of our statistical decision
  process
\begin{center}
  \begin{tabular}{lccc}
      & \multicolumn{2}{c}{Decision} \\
Truth & $H_0$ & $H_a$ \\ \hline
$H_0$ & Correctly accept null & Type I error \\
$H_a$ & Type II error & Correctly reject null  \\ \hline
  \end{tabular}
\end{center}
\end{itemize}
\end{frame}

\begin{frame}\frametitle{Discussion}
\begin{itemize}
\item Consider a court of law; the null hypothesis is that the
  defendant is innocent
\item We require evidence to reject the null hypothesis (convict)
\item If we require little evidence, then we would increase the
  percentage of innocent people convicted (type I errors); however we
  would also increase the percentage of guilty people convicted
  (correctly rejecting the null)
\item If we require a lot of evidence, then we increase the
  percentage of innocent people let free (correctly accepting the
  null) while we would also increase the percentage of guilty people
  let free (type II errors)
\end{itemize}
\end{frame}

\begin{frame}\frametitle{Example}
\begin{itemize}
\item Consider our example again
\item A reasonable strategy would reject the null hypothesis if
  $\bar X$ was larger than some constant, say $C$
\item Typically, $C$ is chosen so that the probability of a Type I
  error, $\alpha$, is $.05$ (or some other relevant constant)
\item $\alpha$ = Type I error rate = Probability of rejecting the null hypothesis
  when, in fact, the null hypothesis is correct
\end{itemize}
\end{frame}

\begin{frame}\frametitle{Continued}
\begin{itemize}
\item Note that 
  \begin{eqnarray*}
    .05 & = & P\left(\bar X \geq C ~|~ \mu = 30 \right) \\ \\
        & = & P\left(\frac{\bar X - 30}{10 / \sqrt{100}} \geq \frac{C - 30}{10/\sqrt{100}} ~|~ \mu = 30\right) \\ \\
        & = & P\left(Z \geq \frac{C - 30}{1}\right)
  \end{eqnarray*}
\item Hence $(C - 30) / 1 = 1.645$ implying $C = 31.645$
\item Since our mean is $32$ we reject the null hypothesis
\end{itemize}
\end{frame}


\begin{frame}\frametitle{Discussion}
\begin{itemize}
\item In general we don't convert $C$ back to the original scale
\item We would just reject because the Z-score; which is how many
  standard errors the sample mean is above the hypothesized mean
  $$
  \frac{32 - 30}{10 / \sqrt{100}} = 2
  $$
  is greater than $1.645$
\end{itemize}

\section{General rules}
\end{frame}
\begin{frame}\frametitle{General rule}
\begin{itemize}
\item The $Z$ test for $H_0:\mu = \mu_0$ versus 
  \begin{itemize}
  \item $H_1: \mu < \mu_0$
  \item $H_2: \mu \neq \mu_0$
  \item $H_3: \mu > \mu_0$ 
  \end{itemize}
 \item Test statistic $ TS = \frac{\bar{X} - \mu_0}{S / \sqrt{n}} $
 \item Reject the null hypothesis when 
  \begin{enumerate}[$H_1:$]
  \item $TS \leq -Z_{1 - \alpha}$
  \item $|TS| \geq Z_{1 - \alpha / 2}$
  \item $TS \geq Z_{1 - \alpha}$
  \end{enumerate}
\end{itemize}

\end{frame}
\begin{frame}\frametitle{Notes}
\begin{itemize}
\item We have fixed $\alpha$ to be low, so if we reject $H_0$ (either
  our model is wrong) or there is a low probability that we have made
  an error
\item We have not fixed the probability of a type II error, $\beta$;
  therefore we tend to say ``Fail to reject $H_0$'' rather than
  accepting $H_0$
\item Statistical significance is not the same as scientific
  significance
\item The region of TS values for which you reject $H_0$ is called the
  rejection region
\end{itemize}
\end{frame}

\section{Notes}
\begin{frame}\frametitle{More notes}
\begin{itemize}
\item The $Z$ test requires the assumptions of the CLT and for $n$ to be large enough
  for it to apply
\item If $n$ is small, then a Gossett's $T$ test is performed exactly in the same way,
  with the normal quantiles replaced by the appropriate Student's $T$ quantiles and
  $n-1$ df
\item The probability of rejecting the null hypothesis when it is false is called {\bf power}
\item Power is used a lot to calculate sample sizes for experiments
\end{itemize}
\end{frame} 


\begin{frame}\frametitle{Example reconsidered}
Consider our example again. Suppose that $n= 16$ (rather than
$100$). Then consider that \\
$$
.05 = P\left(\frac{\bar X - 30}{s / \sqrt{16}} \geq t_{1-\alpha, 15} ~|~ \mu = 30 \right)
$$ \ \\
So that our test statistic is now $\sqrt{16}(32 - 30) / 10 = 0.8 $, while the critical
value is $t_{1-\alpha, 15} = 1.75$. We now fail to reject.
\end{frame}

\section{Two sided tests}
\begin{frame}\frametitle{Two sided tests}
\begin{itemize}
\item Suppose that we would reject the null hypothesis if in fact the 
  mean was too large or too small
\item That is, we want to test the alternative $H_a : \mu \neq 30$
  (doesn't make a lot of sense in our setting)
\item Then note
  \begin{eqnarray*}
\alpha = P\left(\left. \left|\frac{\bar X - 30}{s /\sqrt{16}}\right| > t_{1-\alpha/2,15} ~\right|~ \mu = 30\right)
  \end{eqnarray*}
\item That is we will reject if the test statistic, $0.8$, is either
  too large or too small, but the critical value is calculated using
  $\alpha / 2$
\item In our example the critical value is $2.13$, so we fail to reject.
\end{itemize}
\end{frame}

\section{Confidence intervals}
\begin{frame}\frametitle{Connections with confidence intervals}
\begin{itemize}
\item Consider testing $H_0: \mu = \mu_0$ versus $H_a: \mu \neq \mu_0$
\item Take the set of all possible values for which you fail to reject $H_0$, this
  set is a $(1-\alpha)100\%$ confidence interval for $\mu$
\item The same works in reverse; if a $(1-\alpha)100\%$ interval
  contains $\mu_0$, then we {\bf fail  to} reject $H_0$
\end{itemize}
\end{frame}

\begin{frame}\frametitle{Proof}
\begin{itemize}
\item Consider that we do not reject $H_0$ if
  $$\left| \frac{\bar X - \mu_0}{s /\sqrt{n}} \right| \leq t_{1-\alpha/2, n-1}$$
implying
  $$
  \left|\bar X - \mu_0 \right| \leq t_{1-\alpha/2, n-1} s /\sqrt{n}
  $$
implying
  $$
  \bar X - t_{1-\alpha/2, n-1} s /\sqrt{n} < \mu_0 
  < \bar X + t_{1-\alpha/2, n-1} s /\sqrt{n} 
  $$
\end{itemize}
\end{frame}

\section{P-values}
\begin{frame}\frametitle{P-values}
\begin{itemize}
\item Notice that we rejected the one sided test when $\alpha = 0.05$, would we reject
  if $\alpha = 0.01$, how about $0.001$?
\item The smallest value for alpha that you still reject the null hypothesis is called the {\bf attained significance level}
\item This is equivalent, but philosophically a little different from, the {\bf P-value}
\item The P-value is the probability under the null hypothesis of
  obtaining evidence as extreme or more extreme than would be observed
  by chance alone
\item If the P-value is small, then either $H_0$ is true and we have
  observed a rare event or $H_0$ is false
\end{itemize}
\end{frame} 

\begin{frame}[fragile]\frametitle{Example} 
In our example the $T$ statistic was $0.8$. What's the probability of getting
a $T$ statistic as large as $0.8$?
\begin{verbatim}
pt(0.8, 15, lower.tail = FALSE) ##works out to be 0.22
\end{verbatim}
Therefore, the probability of seeing evidence as extreme or more
extreme than that actually obtained is $0.22$
\end{frame}

\begin{frame}\frametitle{Notes}
\begin{itemize}
\item By reporting a P-value the reader can perform the hypothesis
  test at whatever $\alpha$ level he or she choses
\item If the P-value is less than $\alpha$ you reject the null hypothesis 
\item For two sided hypothesis test, double the smaller of the two one
  sided hypothesis test Pvalues
\item Don't just report P-values, give CIs too!
\end{itemize}
\end{frame}

\begin{frame}\frametitle{Criticisms of the P-value}
\begin{itemize}
\item P-values only consider significance, unlike CIs
\item It is difficult with a P-value or result of a hypothesis test to
  distinguish practical significance from statistical significance
\item Absolute measures of the rareness of an event are not good
  measures of the evidence for or against a hypothesis
\item P-values have become abusively used
\end{itemize}
\end{frame}


\end{document}

