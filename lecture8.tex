\documentclass[aspectratio=169]{beamer}
\mode<presentation>
%\usetheme{Warsaw}
%\usetheme{Goettingen}
\usetheme{Hannover}
%\useoutertheme{default}

%\useoutertheme{infolines}
\useoutertheme{sidebar}
\usecolortheme{dolphin}

\usepackage{amsmath}
\usepackage{amssymb}
\usepackage{enumerate}

%some bold math symbols
\newcommand{\Cov}{\mathrm{Cov}}
\newcommand{\Cor}{\mathrm{Cor}}
\newcommand{\Var}{\mathrm{Var}}
\newcommand{\brho}{\boldsymbol{\rho}}
\newcommand{\bSigma}{\boldsymbol{\Sigma}}
\newcommand{\btheta}{\boldsymbol{\theta}}
\newcommand{\bbeta}{\boldsymbol{\beta}}
\newcommand{\bmu}{\boldsymbol{\mu}}
\newcommand{\bW}{\mathbf{W}}
\newcommand{\one}{\mathbf{1}}
\newcommand{\bH}{\mathbf{H}}
\newcommand{\by}{\mathbf{y}}
\newcommand{\bolde}{\mathbf{e}}
\newcommand{\bx}{\mathbf{x}}

\newcommand{\cpp}[1]{\texttt{#1}}

 \title{Mathematical Biostatistics Boot Camp 2: Lecture 8, Chi-Squared Tests}
\author{Brian Caffo}
\date{\today}
\institute[Department of Biostatistics]{
  Department of Biostatistics \\
  Johns Hopkins Bloomberg School of Public Health\\
  Johns Hopkins University
}


\begin{document}
\frame{\titlepage}

%\section{Table of contents}
\frame{
  \frametitle{Table of contents}
  \tableofcontents
}
\section{Chi-squared testing}
\begin{frame}\frametitle{Chi-squared testing}
\begin{itemize}
\item An alternative approach to testing equality of
  proportions uses the chi-squared statistic
  $$
  \sum \frac{(\mbox{Observed} - \mbox{Expected})^2}{\mbox{Expected}}
  $$
\item ``Observed'' are the observed counts
\item ``Expected'' are the expected counts under the null hypothesis
\item The sum is over all four cells
\item This statistic follows a Chi-squared distribution with 1 df
\item The Chi-squared statistic is exactly the square
  of the difference in proportions Score statistic
\end{itemize}
\end{frame}

\begin{frame}\frametitle{Example}
\begin{center}
\begin{tabular}{|c|c|c|c|} \hline
Trt & Side Effects & None & Total \\ \hline
$X$ & $44$  & $56$  &$100$ \\   \hline
$Y$ & $77$  & $43$ & $120$ \\   \hline
    & $121$ & $99$ & $220$ \\   \hline
\end{tabular}
\end{center}
\begin{itemize}
\item $p_1$ and $p_2$ are the rates of side effects.
\item $H_0:p_1 = p_2$
\end{itemize}
\end{frame}

\begin{frame}[fragile]
\begin{itemize}
\item The $\chi^2$ statistic is $\sum \frac{(O - E)^2}{E}$
\item $O_{11} = 44$, $E_{11} = \frac{121}{220}\times 100 = 55$
\item $O_{21} = 77$, $E_{21} = \frac{121}{220}\times 120 = 66$
\item $O_{12} = 56$, $E_{12} = \frac{99}{220}\times 100 = 45$
\item $O_{22} = 43$, $E_{22} = \frac{99}{220}\times 120 = 54$
$$
\chi^2  = \frac{(44 - 55)^2}{55} + \frac{(77 - 66)^2}{666} 
         + \frac{(56 - 45)^2}{45} + \frac{(43 - 54)^2}{54} 
$$
Which turns out to be $8.96$. Compare to a $\chi^2$ with one degree of
freedom (reject for large values).
\begin{verbatim}
pchisq(8.96, 1, lower.tail = FALSE) 
#result is 0.002
\end{verbatim}
\end{itemize}
\end{frame}

\begin{frame}[fragile]\frametitle{R code}
\begin{verbatim}
dat <- matrix(c(44, 77, 56, 43), 2)
chisq.test(dat)
chisq.test(dat, correct = FALSE)
\end{verbatim}  
\end{frame}

\begin{frame}\frametitle{Notation reminder}
\begin{center}
\begin{tabular}{|c|c|c|}\hline
$n_{11} = x$ & $n_{12} = n_1 - x$ & $n_1 = n_{1+}$ \\ \hline
$n_{21} = y$ & $n_{22} = n_2 - y$ & $n_2 = n_{2+}$ \\ \hline
$n_{+1}$     & $n_{+2}$           &   $n$    \\ \hline 
\end{tabular}
\end{center}
\end{frame}

\begin{frame}\frametitle{Notes}
\begin{itemize}
\item Reject if the statistic is too large
\item Alternative is two sided
\item Do not divide $\alpha$ by $2$
\item A small $\chi^2$ statistic implies little difference between the
  observed values and those expected under $H_0$
\item The $\chi^2$ statistic and approach generalizes to other kinds of
  tests and larger contingency tables
\item Alternative computational form for the $\chi^2$ statistic
$$
\chi^2 = \frac{n(n_{11} n_{22} - n_{12}n_{21})^2}{n_{+1} n_{+2} n_{1+} n_{2+}}
$$
\end{itemize}
\end{frame} 

\begin{frame}\frametitle{Notes}
\begin{itemize}
\item Notice that the statistic:
$$
\chi^2 = \frac{n(n_{11} n_{22} - n_{12}n_{21})^2}{n_{+1} n_{+2} n_{1+} n_{2+}}
$$
does not change if you transpose the rows and the columns of the table
\item Surprisingly, the $\chi^2$ statistic can be used 
  \begin{itemize}
  \item the rows are fixed (binomial)
  \item the columns are fixed (binomial)
  \item the total sample size is fixed (multinomial)
  \item none are fixed (Poisson) 
  \end{itemize}
\item For a given set of data, any of these assumptions results
  in the same value for the statistic
\end{itemize}
\end{frame}

\section{Testing independence}
\begin{frame}\frametitle{Testing independence}
\begin{itemize}
\item Maternal age versus birthweight\footnote{From Agresti Categorical Data Analysis second edition}
\item Cross-sectional sample, only the total sample size is fixed
\end{itemize}
\begin{center}
\begin{tabular}{|c|c|c|c|} \hline
          & \multicolumn{2}{c}{Birthweight} & \\ \hline
Mat. Age  & $<2500g$ & $\geq 2,500g$ & Total \\ \hline
$< 20 y$    & $20$   & $80$  & $100$ \\   
$\geq 20 y$ & $30$   & $270$ & $300$ \\   \hline
Total       & $50$   & $350$ & $400$ \\   \hline
\end{tabular}
\end{center}
\begin{itemize}
\item $H_0:$ MA is independent of BW
\item $H_a:$ MA is not independent of BW
\end{itemize}
\end{frame}

\begin{frame}\frametitle{Continued}
\begin{itemize}
\item Estimated marginal probability of younger maternal age $P\left(\mbox{MA}<20\right) = \frac{100}{400} = .25$
\item Estimated marginal probability of low birth weight $P\left(\mbox{BW}<2500\right) = \frac{50}{400} = .125$
\item Under $H_0$ estimated cell probability of younger and low birth weight 
  $$P\left(\mbox{MA}<20\mbox{ and BW}<2,500 \right) = .25 \times .125$$
\item Therefore
  \begin{itemize}
  \item $E_{11} = \frac{100}{400}\times\frac{50}{400}\times 400 = 12.5$
  \item $E_{12} = \frac{100}{400}\times\frac{350}{400}\times 400 = 87.5$
  \item $E_{21} = \frac{300}{400}\times\frac{50}{400}\times 400 = 37.5$
  \item $E_{22} = \frac{300}{400}\times\frac{350}{400}\times 400 = 262.5$
  \item $\chi^2 = \frac{(20 - 12.5)^2}{12.5} + \frac{(80- 87.5)^2}{87.5}
                  \frac{(30 - 37.5)^2}{37.5} + \frac{(270- 262.5)^2}{262.5}= 6.86$
  \end{itemize}
\item Compare to critical value \\
 \texttt{qchisq(.95, 1)=3.84}
\item Or calculate P-value \\ 
  \texttt{pchisq(6.86, 1, lower.tail = F)=.009}
\end{itemize}
\end{frame}

\section{Testing equality of several proportions}
\begin{frame}\frametitle{Chi-squared testing cont'd}
\ttfamily
\begin{center}
\begin{tabular}{llll}
& \multicolumn{2}{c}{Alcohol use} & \\
Group      & High & Low & Total \\ \hline
Clergy     & 32   & 268 & 300 \\
Educators  & 51   & 199 & 250 \\
Executives & 67   & 233 & 300 \\
Retailers  & 83   & 267 & 350 \\ \hline
Total      & 233  & 967 & 1,200
\end{tabular}
\end{center}
\normalfont
~\footnote{From Agresti's Categorical Data Analysis second edition}
\end{frame}

\begin{frame}
\begin{itemize}
\item Interest lies in testing whether or not the proportion of high alcohol use
  is the same in the four occupations
\item $H_0:p_1 = p_2 = p_3 = p_4 = p$ 
\item $H_a:$ at least two of the $p_j$ are unequal 
\item $O_{11} = 32$, $E_{11} = 300 \times \frac{233}{1200}$
\item $O_{12} = 268$, $E_{12} = 300 \times \frac{967}{1200}$
\item ...
\item Chi-squared statistic $\sum \frac{(0 - E)^2}{E} = 20.59$
\item $df=(Rows - 1)(Columns - 1) = 3$
\item Pvalue \texttt{pchisq(20.59, 3, lower.tail = FALSE)}$\approx 0$
\end{itemize}
\end{frame} 

\section{Generalization}
\begin{frame}\frametitle{Word distributions}
\begin{center}
\ttfamily
  \begin{tabular}{llllll}
       & \multicolumn{4}{c}{Book} &\\
Word      & 1   & 2   & 3   &  Total\\ \hline
$a$       & 147 & 186 & 101 &  434\\
$an$      & 25  & 26  & 11  &  62\\
$this$    & 32  & 39  & 15  &  86\\
$that$    & 94  & 105 & 37  &  236\\
$with$    & 59  & 74  & 28  &  161\\
$without$ & 18  & 10  & 10  &  38\\ \hline
Total     & 375 & 440 & 202 &  1017\\
  \end{tabular}
\normalfont
\end{center}
~\footnote{From Rice Mathematical Statistics and Data Analysis, second edition}
\end{frame}

\begin{frame}
\begin{itemize}
\item $H_0:$ The probabilities of each word \\
  are the same for every book
\item $H_a:$ At least two are different 
\item $O_{11} = 147$ $E_{11} = 375 \times \frac{434}{1017}$
\item $O_{12} = 186$ $E_{12} = 440 \times \frac{434}{1017}$
\item ...
\item $\sum \frac{(O - E)^2}{E} = 12.27$
\item $df = (6 - 1) (3 - 1) = 10$ 
\end{itemize}
\end{frame}

\section{Independence}
\begin{frame}\frametitle{Testing independence}
\ttfamily
\begin{center}
\begin{tabular}{clllll}  
& \multicolumn{4}{c}{Wife's Rating} \\ 
Husband & N & F & V & A & Tot               \\ \hline
N & 7  &7  & 2  & 3  & 19 \\ 
F & 2  &8  & 3  & 7  & 20 \\ 
V & 1  &5  & 4  & 9  & 19 \\ 
A & 2  &8  & 9  & 14 & 33 \\ \hline
  & 12 &28 & 18 & 33 & 91 \\
\end{tabular}
\end{center}
N=never, F=fairly often, V=very often, A=almost always \\
~\footnote{From Agresti's Categorical Data Analysis second edition}
\normalfont
\end{frame}

\begin{frame}\frametitle{Independence cont'd}
\begin{itemize}
\item $H_0:$ H and W ratings are independent
\item $H_a:$ not independent
\item $P(H = N ~\&~ W = A) = P(H = N)P(W = A)$ 
\item $stat = \sum \frac{(O - E)^2}{E}$
\item $O_{11} = 7$ $E_{11} = 91\times\frac{19}{91}\times\frac{12}{91} = 2.51$
\item $E_{ij} = n_{i+}n_{+j}/n$
\item $df = (Rows - 1)(Cols - 1)$
\end{itemize}
\end{frame}

\begin{frame}[fragile]\frametitle{Independence cont'd}
\begin{verbatim}
x<-matrix(c(7,7,2,3,
            2,8,3,7,
            1,5,4,9,
            2,8,9,14),4)
chisq.test(x)
\end{verbatim} 
\begin{itemize}
\item $\sum \frac{(O - E)^2}{E} = 16.96$
\item $df = (4 - 1)(4 - 1) = 9$ 
\item $p-value = .049$
\item Cell counts might be too small to use
large sample approximation
\end{itemize}
\end{frame}

\begin{frame}\frametitle{Notes}
\begin{itemize}
\item Equal distribution and independence test yield the same results
\item Same test results if
  \begin{itemize}
  \item row totals are fixed 
  \item column totals are fixed
  \item total ss is fixed
  \item none are fixed
  \end{itemize}
\item Note that this is common in statistics; mathematically
  equivalent results are applied in different settings, but result in
  different interpretations
\end{itemize}
\end{frame}

\begin{frame}
\begin{itemize}
\item  Chi-squared result requires large cell counts
\item $df$ is always $(Rows - 1)(Columns - 1)$
\item Generalizations of Fishers exact test can be used or continuity
  corrections can be employed
\end{itemize}
\end{frame}

\section{Monte Carlo}
\begin{frame}[fragile]\frametitle{Exact permutation test}
\begin{itemize}
\item Reconstruct the individual data
\begin{verbatim}
W:NNNNNNNFFFFFFFVVAAANNFFFFFFFF ... 
H:NNNNNNNNNNNNNNNNNNNFFFFFFFFFF ...
\end{verbatim}
\item Permute either the \texttt{W} or \texttt{H} row
\item Recalculate the contingency table
\item Calculate the $\chi^2$ statistic for each permutation
\item Percentage of times it is larger than the observed value is an
  exact P-value
\begin{verbatim}
chisq.test(x, simulate.p.value = TRUE)
\end{verbatim}
\end{itemize}
\end{frame}

\begin{frame}[fragile]\frametitle{Chi-squared goodness of fit}
Results from R's RNG 
\begin{center}
\ttfamily
\begin{tabular}{llllll}
      & $[0,.25)$ & $[.25, .5)$ & $[.5, .75)$ & $[.75, 1)$ & Total \\ \hline
Count & 254 & 235 & 267 & 244 & 1000 \\
TP    & .25 & .25 & .25 & .25 & 1 \\ \hline
\end{tabular}
\normalfont
\end{center}
\begin{itemize}
\item $H_0:~p_1 = .25,~p_2 = .25,~ p_3 = .25,~ p_4 = .25$
\item $H_a:$ any $p_i \neq$ it's hypothesized value 
\end{itemize}
\end{frame}

\begin{frame}\frametitle{Continued}
\begin{itemize}
\item $O_1 = 254$ $E_1 = 1000 \times .25 = 250$
\item $O_2 = 235$ $E_2 = 1000 \times .25 = 250$
\item $O_3 = 267$ $E_3 = 1000 \times .25 = 250$
\item $O_4 = 244$ $E_4 = 1000 \times .25 = 250$
\item $\sum \frac{(O - E)^2}{E} = 2.264$
\item $df = 3$
\item $P-value = .52$
\end{itemize}
\end{frame}

\section{Goodness of fit testing}
\begin{frame}\frametitle{Testing Mendel's hypothesis}
\begin{center}  
\ttfamily
\begin{tabular}{llll}
                 & \multicolumn{2}{c}{Phenotype} \\
                 & Yellow  & Green   & Total \\ \hline
Observed         & 6022    & 2001    & 8023\\
TP               & .75     & .25     & 1   \\ 
Expected         & 6017.25 & 2005.75 & 8023\\ \hline 
\end{tabular}
\normalfont
\end{center}
\begin{itemize}
\item $H_0:p_1 = .75, ~ p_2 = .25$ 
\item $\sum \frac{(0-E)^2}{E} = \frac{(6022 - 6017.25)^2}{6017.25} +
  \frac{(2001 - 2005.75)^2}{2005.75} = .015$
\end{itemize}
\end{frame}

\begin{frame}\frametitle{Continued}
\begin{itemize}
\item $df = 1$
\item P-value $= .90$
\item Fisher combined several of Mendel's tables
\item $\sum \chi^2_{v_i} \sim \chi^2_{\sum v_i}$
\item Statistic $42$, $df = 84$, P-value $= .99996$
\item Agreement with theoretical counts is perhaps too good?
\end{itemize}
\end{frame}

\begin{frame}\frametitle{Notes on GOF}
\begin{itemize}
\item Test of whether or not observed 
  counts equal theoretical values
\item Test statistic is $\sum \frac{(0-E)^2}{E}$
\item TS follows $\chi^2$ distribution for large $n$
\item $df$ is the number of cells minus 1
\item Especially useful for testing RNGs
\item Kolmogorov/Smirnov test is an  alternative 
  test that does not require  discretization 
\end{itemize}
\end{frame}

\end{document}
