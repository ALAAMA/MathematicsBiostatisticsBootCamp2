\documentclass[aspectratio=169]{beamer}
\mode<presentation>
%\usetheme{Warsaw}
%\usetheme{Goettingen}
\usetheme{Hannover}
%\useoutertheme{default}

%\useoutertheme{infolines}
\useoutertheme{sidebar}
\usecolortheme{dolphin}

\usepackage{amsmath}
\usepackage{amssymb}
\usepackage{enumerate}

%some bold math symbosl
\newcommand{\Cov}{\mathrm{Cov}}
\newcommand{\Var}{\mathrm{Var}}
\newcommand{\brho}{\boldsymbol{\rho}}
\newcommand{\bSigma}{\boldsymbol{\Sigma}}
\newcommand{\btheta}{\boldsymbol{\theta}}
\newcommand{\bbeta}{\boldsymbol{\beta}}
\newcommand{\bmu}{\boldsymbol{\mu}}
\newcommand{\bW}{\mathbf{W}}
\newcommand{\one}{\mathbf{1}}
\newcommand{\bH}{\mathbf{H}}
\newcommand{\by}{\mathbf{y}}
\newcommand{\bolde}{\mathbf{e}}
\newcommand{\bx}{\mathbf{x}}

\newcommand{\cpp}[1]{\texttt{#1}}

\title{Mathematical Biostatistics Boot Camp 2: Lecture 2, Power}
\author{Brian Caffo}
\date{\today}
\institute[Department of Biostatistics]{
  Department of Biostatistics \\
  Johns Hopkins Bloomberg School of Public Health\\
  Johns Hopkins University
}


%\logo{\includegraphics[height=0.5cm]{Logo_PPT.pdf}}

\begin{document}
\frame{\titlepage}

%\section{Table of contents}
\frame{
  \frametitle{Table of contents}
  \tableofcontents
}


\section{Power}
\begin{frame}\frametitle{Power}
\begin{itemize}
\item Power is the probability of rejecting the null hypothesis when it is false
\item Ergo, power (as it's name would suggest) is a good thing; you want more power
\item A type II error (a bad thing, as its name would suggest) is failing to reject the null hypothesis when it's false; the probability of a type II error is usually called $\beta$
\item Note Power  $= 1 - \beta$
\end{itemize}
\end{frame}

\begin{frame}\frametitle{Notes}
\begin{itemize}
\item Consider our previous example involving RDI
\item $H_0: \mu = 30$ versus $H_a: \mu > 30$
\item Then power is 
$$P\left(\frac{\bar X - 30}{s /\sqrt{n}} > t_{1-\alpha,n-1} ~|~ \mu = \mu_a \right)$$
\item Note that this is a function that depends on the specific value of $\mu_a$!
\item Notice as $\mu_a$ approaches $30$ the power approaches $\alpha$
\end{itemize}
\end{frame}

\section{Calculating power}
\begin{frame}\frametitle{Calculating power} 
Assume that $n$ is large and that we know $\sigma$
\begin{eqnarray*}
1 -\beta & = & 
P\left(\frac{\bar X - 30}{\sigma /\sqrt{n}} > z_{1-\alpha} ~|~ \mu = \mu_a \right)\\ \\
& = & P\left(\frac{\bar X - \mu_a + \mu_a - 30}{\sigma /\sqrt{n}} > z_{1-\alpha} ~|~ \mu = \mu_a \right)\\ \\
& = & P\left(\frac{\bar X - \mu_a}{\sigma /\sqrt{n}} > z_{1-\alpha} - \frac{\mu_a - 30}{\sigma /\sqrt{n}} ~|~ \mu = \mu_a \right)\\ \\
& = & P\left(Z > z_{1-\alpha} - \frac{\mu_a - 30}{\sigma /\sqrt{n}} ~|~ \mu = \mu_a \right)\\ \\
\end{eqnarray*}
\end{frame}

\begin{frame}\frametitle{Example continued}
\begin{itemize}
\item Suppose that we wanted to detect a increase in mean RDI
  of at least 2 events / hour (above 30). Assume normality and
  that the sample in question will have a standard deviation of $4$;
  what would be the power if we took a sample size of $16$? \\
\item $Z_{1-\alpha} = 1.645$ and $\frac{\mu_a - 30}{\sigma /\sqrt{n}} = 2 / (4 /\sqrt{16}) = 2$ \\
\item $P(Z > 1.645 - 2) = P(Z > -0.355) = 64\%$
\end{itemize}
\end{frame}

\begin{frame}\frametitle{Example continued}
\begin{itemize}
\item What $n$ would be required to get a power of 80\%
\item I.e. we want
$$0.80 = P\left(Z > z_{1-\alpha} - \frac{\mu_a - 30}{\sigma /\sqrt{n}} ~|~ \mu = \mu_a \right)$$
\item Set $z_{1-\alpha} - \frac{\mu_a - 30}{\sigma /\sqrt{n}} = z_{0.20}$ and solve for $n$
\end{itemize}
\end{frame}

\begin{frame}\frametitle{Notes}
\begin{itemize}
\item The calculation for $H_a:\mu < \mu_0$ is similar
\item For $H_a: \mu \neq \mu_0$ calculate the one sided power using
  $\alpha / 2$ (this is only approximately right, it excludes the probability of
  getting a large TS in the opposite direction of the truth)
\item Power goes up as $\alpha$ gets larger
\item Power of a one sided test is greater than the power of the
  associated two sided test
\item Power goes up as $\mu_1$ gets further away from $\mu_0$
\item Power goes up as $n$ goes up
\end{itemize}
\end{frame}


\section{T-tests}
\begin{frame}\frametitle{Power for the T test}
\begin{itemize}
\item Consider calculating power for a Gossett's $T$ test for our example
\item The power is
  $$
  P\left(\frac{\bar X - 30}{S /\sqrt{n}} > t_{1-\alpha, n-1} ~|~ \mu = \mu_a \right)
  $$
\item Notice that this is equal to
  \begin{eqnarray*}
& = &     
  P\left(\sqrt{n}(\bar X - 30) > t_{1 - \alpha, n-1} S ~|~ \mu = \mu_a \right)\\ \\
& = &     
  P\left(\frac{\sqrt{n}(\bar X - 30)}{\sigma}   > t_{1-\alpha, n-1} \frac{S}{\sigma} ~|~ \mu = \mu_a \right)
  \end{eqnarray*}
\end{itemize}
\end{frame}

\begin{frame}\frametitle{Continued}
\begin{itemize}
\item Continued
$$
  P\left(\frac{\sqrt{n}(\bar X - \mu_a)}{\sigma} + \frac{\sqrt{n}(\mu_a - 30)}{\sigma} > \frac{t_{1-\alpha, n-1}}{\sqrt{n-1}}\times \sqrt{\frac{(n-1) S^2}{\sigma^2}} \right)$$
(where we ommitted the conditional on $\mu_a$ part for space)
\item This is now equal to
$$
P\left(Z + \frac{\sqrt{n}(\mu_a - 30)}{\sigma} >  \frac{t_{1 - \alpha, n-1}}{\sqrt{n-1}} \sqrt{\chi^2_{n-1}}\right)
$$
where $Z$ and $\chi^2_{n-1}$ are independent standard normal and chi-squared random variables
\item While computing this probability is outside the scope of the class, it would be easy to approximate with Monte Carlo
\end{itemize}
\end{frame}


\begin{frame}[fragile]\frametitle{Example}
Let's recalculate power for the previous example using the $T$ distribution
instead of the normal; here's the easy way to do it. Let $\sigma = 4$
and $\mu_a - \mu_0 = 2$
\begin{verbatim}
##the easy way
power.t.test(n = 16, delta = 2 / 4, 
             type = "one.sample", 
             alt = "one.sided")
##result is 60%
\end{verbatim}
\end{frame}

\section{Monte Carlo}
\begin{frame}[fragile]\frametitle{Example}
Using Monte Carlo
\begin{verbatim}
nosim <- 100000
n <- 16
sigma <- 4
mu0 <- 30
mua <- 32
z <- rnorm(nosim)
xsq <- rchisq(nosim, df = 15)
t <- qt(.95, 15)
mean(z + sqrt(n) * (mua - mu0) / sigma >
     t / sqrt(n - 1) * sqrt(xsq))
##result is 60%
\end{verbatim}
\end{frame}

\begin{frame}\frametitle{Comments}
\begin{itemize}
\item Notice that in both cases, power required a true mean and a true standard deviation
\item However in this (and most linear models) the power depends only on the mean (or change in means) divided by the standard deviation
\end{itemize}
\end{frame}



\end{document}

