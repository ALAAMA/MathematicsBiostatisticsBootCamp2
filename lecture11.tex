\documentclass[aspectratio=169]{beamer}
\mode<presentation>
%\usetheme{Warsaw}
%\usetheme{Goettingen}
\usetheme{Hannover}
%\useoutertheme{default}

%\useoutertheme{infolines}
\useoutertheme{sidebar}
\usecolortheme{dolphin}

\usepackage{amsmath}
\usepackage{amssymb}
\usepackage{enumerate}

%some bold math symbosl
\newcommand{\Cov}{\mathrm{Cov}}
\newcommand{\Cor}{\mathrm{Cor}}
\newcommand{\Var}{\mathrm{Var}}
\newcommand{\brho}{\boldsymbol{\rho}}
\newcommand{\bSigma}{\boldsymbol{\Sigma}}
\newcommand{\btheta}{\boldsymbol{\theta}}
\newcommand{\bbeta}{\boldsymbol{\beta}}
\newcommand{\bmu}{\boldsymbol{\mu}}
\newcommand{\bW}{\mathbf{W}}
\newcommand{\one}{\mathbf{1}}
\newcommand{\bH}{\mathbf{H}}
\newcommand{\by}{\mathbf{y}}
\newcommand{\bolde}{\mathbf{e}}
\newcommand{\bx}{\mathbf{x}}
\newcommand{\cpp}[1]{\texttt{#1}}
\newcommand{\logit}{\mathrm{logit}}

\title{Mathematical Biostatistics Boot Camp 2: 
Lecture 11, Matched Two by Two Tables}
\author{Brian Caffo}
\date{\today}
\institute[Department of Biostatistics]{
  Department of Biostatistics \\
  Johns Hopkins Bloomberg School of Public Health\\
  Johns Hopkins University
}


\begin{document}
\frame{\titlepage}

%\section{Table of contents}
\frame{
  \frametitle{Table of contents}
  \tableofcontents
}

\section{Matched pairs data}
\begin{frame}\frametitle{Matched pairs binary data}
\begin{center}
\ttfamily
  \begin{tabular}{lccc}
First & \multicolumn{2}{c}{Second Survey} & \\ \cline{2-3}
survey & Approve & Disapprove & Total \\ \hline
Approve & 794 & 150 & 944 \\
Disapprove & 86 & 570 & 656 \\
Total & 880 & 720 & 1600 \\ \hline
  \end{tabular}
\normalfont
\end{center}
\begin{center}
\ttfamily
  \begin{tabular}{lccc}
          & \multicolumn{2}{c}{Cases} & \\ \cline{2-3}
Controls  & Exposed & Unexposed & Total \\ \hline
Exposed   & 27      & 29        & 56  \\
Unexposed &  3      &  4        &  7  \\
Total     & 30      &  33       & 63   \\ \hline
  \end{tabular}
\normalfont
\end{center}
~\footnote{Both data sets from Agresti, Categorical Data Analysis, second edition}
\end{frame}

\section{Dependence}
\begin{frame}\frametitle{Dependence}
\begin{itemize}
\item Matched binary can arise from
  \begin{itemize}
  \item Measuring a response at two occasions
  \item Matching on case status in a retrospective study
  \item Matching on exposure status in a prospective or cross-sectional study
  \end{itemize}
\item The pairs on binary observations are dependent, so our
  existing methods do not apply
\item We will discuss the process of making conclusions about
  the marginal probabilities and odds
\end{itemize}
\end{frame}

\begin{frame}\frametitle{Notation}
  \begin{center}
\ttfamily
    \begin{tabular}{cccc} 
   & \multicolumn{2}{c}{time 2} & \\
time 1 & Yes & No & Total\\
  Yes  & $n_{11}$ & $n_{12}$ & $n_{1+}$\\
   no  & $n_{21}$ & $n_{22}$ & $n_{2+}$\\
Total  & $n_{+1}$ & $n_{+2}$ & $n$
    \end{tabular}
    \begin{tabular}{cccc} 
   & \multicolumn{2}{c}{time 2} & \\
time 1 & Yes & No & Total\\
  Yes  & $\pi_{11}$ & $\pi_{12}$ & $\pi_{1+}$\\
   no  & $\pi_{21}$ & $\pi_{22}$ & $\pi_{2+}$\\
Total  & $\pi_{+1}$ & $\pi_{+2}$ & $1$
    \end{tabular}
  \end{center}
  \begin{itemize}
  \item We assume that the $(n_{11},n_{12},n_{21},n_{22})$ are multinomial
    with $n$ trials and probabilities $(\pi_{11},\pi_{12},\pi_{21},\pi_{22})$
  \item $\pi_{1+}$ and $\pi_{+1}$ are the marginal
    probabilities of a yes response at the two occasions
  \item $\pi_{1+} = P(\mbox{Yes} ~|~ \mbox{Time 1})$
  \item $\pi_{+1} = P(\mbox{Yes} ~|~ \mbox{Time 2})$
  \end{itemize}
\end{frame}

\section{Marginal homogeneity}
\begin{frame}\frametitle{Marginal homogeneity}
\begin{itemize}
\item Marginal homogeneity is the hypothesis $H_0:\pi_{1+} = \pi_{+1}$
\item Marginal homogeneity is equivalent to symmetry
  $H_0:\pi_{12} = \pi_{21}$ 
\item The obvious estimate of  $\pi_{12} - \pi_{21}$ is 
  $n_{12}/n - n_{21}/n$
\item Under $H_0$ a consistent estimate of the variance is
   $(n_{12} + n_{21}) / n ^ 2$ 
\item Therefore 
  $$
  \frac{(n_{12} - n_{21})^2}{n_{12} + n_{21}}
  $$
follows an asymptotic $\chi^2$ distribution with 1 degree of freedom
\end{itemize}
\end{frame}

\section{McNemar's test}
\begin{frame}\frametitle{McNemar's test}
\begin{itemize}
\item The test from the previous page is called McNemar's test
\item Notice that only the discordant cells enter into the test
  \begin{itemize}
  \item $n_{12}$ and $n_{21}$ carry the relevant information about whether
    or not $\pi_{1+}$ and $\pi_{+1}$ differ
  \item $n_{11}$ and $n_{22}$ contribute information to estimating the
    magnitude of this difference
  \end{itemize}
\end{itemize}
\end{frame}

\begin{frame}[fragile]\frametitle{Example}
\begin{itemize}
\item Test statistic $\frac{(86 - 150)^2}{86 + 150} = 17.36$
\item P-value = $3\times 10^{-5}$
\item Hence we reject the null hypothesis and conclude that there is
evidence to suggest a change in opinion
between the two polls
\item In R
\begin{verbatim}
mcnemar.test(matrix(c(794, 86, 150, 570), 2), 
             correct = FALSE)
\end{verbatim}
The \texttt{correct} option applies a continuity correction
\end{itemize}
\end{frame}
 
\section{Estimation}
\begin{frame}\frametitle{Estimation}
\begin{itemize}
\item Let $\hat \pi_{ij} = n_{ij} / n$ be the sample proportions
\item $d = \hat \pi_{1+} - \hat \pi_{+1} = (n_{12} - n_{21})/n$ estimates the difference
  in the marginal proportions
\item The variance of $d$ is 
$$ 
  \sigma_d^2 = \{\pi_{1+}(1 - \pi_{1+}) + \pi_{+1}(1 - \pi_{+1}) - 2 (\pi_{11}\pi_{22} - \pi_{12}\pi_{21})\}/n
$$
\item $\frac{d - (\pi_{1+} - \pi_{+1})}{\hat \sigma_d}$ follows an asymptotic normal distribution
\item Compare $\sigma_d^2$ with what we would use if the proportions were independent
\end{itemize}
\end{frame}

\begin{frame}\frametitle{Example}
\begin{itemize}
\item $d = 944 / 1600 - 880 / 1600 = .59 - .55 = .04$ 
\item $\hat \pi_{11} = .50$, $\hat \pi_{12} =.09$, $\hat \pi_{21} = .05$, $\hat \pi_{22} = .36$ 
\item $\hat \sigma^2_d = \{.59(1-.59)+.55(1-.55) - 2(.50\times .36 - .09\times .05)\} / 1600$ 
\item $\hat \sigma_d = .0095$
\item 95\% CI - $.04 \pm 1.96 \times .0095 = [.06, .02]$ 
\item Note ignoring the dependence yields $\hat \sigma_d = .0175$
\end{itemize}
\end{frame}

\section{Relationship with CMH}
\begin{frame}\frametitle{Relationship with CMH test}
\begin{itemize}
\item Each subject's (or matched pair's) responses can be represented as one
of four tables.
  \begin{center}
\ttfamily
    \begin{tabular}{lll}
   & \multicolumn{2}{c}{Response} \\
Time   & Yes      & No  \\
  First      & 1        & 0         \\
  Second   & 1        & 0         \\
    \end{tabular}
    \begin{tabular}{lll}
   & \multicolumn{2}{c}{Response} \\
Time   & Yes      & No  \\
  First      & 1        & 0         \\
  Second   & 0        & 1         \\
    \end{tabular} \\
    \begin{tabular}{lll}
   & \multicolumn{2}{c}{Response} \\
Time   & Yes      & No  \\
  First      & 0        & 1         \\
  Second   & 1        & 0         \\
    \end{tabular}
    \begin{tabular}{lll}
   & \multicolumn{2}{c}{Response} \\
Time   & Yes      & No  \\
  First      & 0        & 1         \\
  Second   & 0        & 1         \\
    \end{tabular}
 \end{center}
\end{itemize}
\end{frame}

\begin{frame}\frametitle{Result}
\begin{itemize}
\item McNemar's test is equivalent to the CMH test where subject is
  the stratifying variable and each 2$\times$2 table is the observed
  zero-one table for that subject
\item This representation is only useful for conceptual purposes
\end{itemize}
\end{frame}


\begin{frame}\frametitle{Exact version}
\begin{itemize}
\item Consider the cells $n_{12}$ and $n_{21}$
\item Under $H_0$, $\pi_{12} / (\pi_{12} + \pi_{21}) = .5$
\item Therefore, under $H_0$, $n_{21} ~|~ n_{21} + n_{12}$ is binomial
  with success probability $.5$ and $n_{21} + n_{12}$ trials
\item We can use this result to come up with an exact P-value for
  matched pairs data
\end{itemize}
\end{frame}

\begin{frame}
\begin{itemize}
\item Consider the approval rating data
\item $H_0 : \pi_{21} = \pi_{12}$ versus $H_a : \pi_{21} < \pi_{12}$ ($\pi_{+1} < \pi_{1+}$)
\item $P(X \leq 86 ~|~ 86 + 150) = .000$ where $X$ is binomial with
  $236$ trials and success probability $p = .5$
\item For two sided tests, double the smaller of the two one-sided
  tests
\end{itemize}
\end{frame}


\section{Marginal odds ratios}
\begin{frame}\frametitle{Estimating the marginal odds ratio}
\begin{itemize}
\item The marginal odds ratio is
$$
\frac{\pi_{1+}/\pi_{2+}}{\pi_{+1}/\pi_{+2}}
  = \frac{\pi_{1+}\pi_{+2}}{\pi_{+1}\pi_{2+}}
$$
\item The maximum likelihood estimate of the margina {\em log} odds ratio is
$$\hat \theta = \log\{\hat \pi_{1+} \hat \pi_{+2}/ \hat \pi_{+1} \hat \pi_{2+}\}$$
\item The asymptotic variance of this estimator is
  \begin{eqnarray*}
&   & \{ (\pi_{1+}\pi_{2+})^{-1} + (\pi_{+1}\pi_{+2})^{-1} \\
& - & 2(\pi_{11}\pi_{22} - \pi_{12}\pi_{21})/ (\pi_{1+}\pi_{2+}\pi_{+1}\pi_{+2})\}/n
  \end{eqnarray*}
\end{itemize}
\end{frame}

\begin{frame}\frametitle{Example}
\begin{itemize}
\item In the approval rating example the marginal OR compares
the odds of approval at time 1 to that at time 2
\item $\hat \theta = \log(944\times 720 / 880 \times 656) = .16$
\item Estimated standard error = $.039$
\item CI for the log odds ratio = $.16 \pm 1.96 \times .039 = [.084, .236]$
\end{itemize}
\end{frame}

\section{Conditional versus marginal}

\begin{frame}\frametitle{Conditional versus marginal odds}
\begin{itemize}
\item $n_{ij}$ cell counts 
\item $n$ total sample size
\item $\pi_{ij}$ the multinomial probabilities
\item The ML estimate of the marginal {\em log} odds ratio is
$$\hat \theta = \log\{\hat \pi_{1+} \hat \pi_{+2}/ \hat \pi_{+1} \hat \pi_{2+}\}$$
\item The asymptotic variance of this estimator is
  \begin{eqnarray*}
&   & \{ (\pi_{1+}\pi_{2+})^{-1} + (\pi_{+1}\pi_{+2})^{-1} \\
& - & 2(\pi_{11}\pi_{22} - \pi_{12}\pi_{21})/ (\pi_{1+}\pi_{2+}\pi_{+1}\pi_{+2})\}/n
  \end{eqnarray*}
\end{itemize}
\end{frame}

\section{Conditional ML}
\begin{frame}\frametitle{Conditional ML}
\begin{itemize}
\item Consider the following model
\begin{eqnarray*}
&&\mathrm{logit}\{P(\mbox{Person} ~i~ \mbox{says Yes at Time 1})\} = \alpha + U_i\\
&&\mathrm{logit}\{P(\mbox{Person} ~i~ \mbox{says Yes at Time 2})\} = \alpha + \gamma + U_i
\end{eqnarray*}
\item Each $U_i$ contains person-specific effects. A person with a large $U_i$ is likely
  to answer Yes at both occasions.
\item $\gamma$ is the {\bf log odds ratio} comparing a response of Yes at Time 1 to 
  a response of Yes at Time 2.
\item $\gamma$ is {\bf subject specific effect}. If you subtract the log odds of a yes response
  for two different people, the $U_i$ terms would not cancel
\end{itemize}
\end{frame}

\begin{frame}\frametitle{Conditional ML cont'd}
\begin{itemize}
\item One way to eliminate the $U_i$ and get a good estimate of $\gamma$ is 
  to condition on the total number of Yes responses for each person
  \begin{itemize}
  \item If they answered Yes or No on both occasions then you know both responses
  \item Therefore, only discordant pairs have any relevant information
    after conditioning
  \end{itemize}
\item The conditional ML estimate for $\gamma$  and its SE turn out to be
$$
\log\{n_{21}/n_{12}\} ~~~~~ \sqrt{1/n_{21} + 1/n_{12}}
$$
\end{itemize}
\end{frame}

\begin{frame}\frametitle{Distinctions in interpretations}
\begin{itemize}
\item The marginal ML has a marginal interpretation. The effect is averaged over
  all of the values of $U_i$. 
\item The conditional ML estimate has a subject specific interpretation. 
\item Marginal interpretations are more useful for policy type statements. Policy
  makers tend to be interested in how factors influence populations.
\item Subject specific interpretations are more useful in clinical applications. 
  Physicians are interested in how factors influence individuals.
\end{itemize}
\end{frame}



\end{document}
